\documentclass[12pt,a4paper]{article}
\usepackage{ctex}
%\usepackage{geometry}
%\bibliographystyle{plain}
\usepackage[super]{gbt7714}
%\geometry{a4paper}
\usepackage[margin=2.3cm]{geometry}
\usepackage{graphicx}
\usepackage{float}%强制左对齐1
\usepackage{longtable}
\usepackage{listings} 

\usepackage{graphicx}
\graphicspath{{figures/}}
%\usepackage{biblatex}
\usepackage{amsmath}
\usepackage{amsfonts}
\usepackage{xcolor}
\usepackage{soul}
\usepackage{booktabs}
\title{\heiti 实验二 \quad ENVI基本操作}
\author{}

\date{ }


\newcommand{\upcite}[1]{\textsuperscript{\textsuperscript{\cite{#1}}}} 
\def\degree{${}^{\circ}$}
\definecolor{Seashell}{RGB}{255, 245, 238} %背景色浅一点的
\definecolor{Firebrick4}{RGB}{255, 0, 0}%文字颜色红一点的
\newcommand{\code}[1]{
	\begingroup
	\sethlcolor{Seashell}%背景色
	\textcolor{Firebrick4}{\hl{#1}}%textcolor里面对应文字颜色
	\endgroup
}

\begin{document}
	
	\maketitle	
\setcounter{page}{5}

\leftline{\textbf{实验时间}:2021年11月26日}	
\leftline{\textbf{实验环境}:电脑:Windows 10(64 bit); \quad 软件: ENVI 5.3 \quad SP1(64bit)}	
	
	\section{实验要求}
		\begin{enumerate}
		
		\item 在ENVI中打开家乡影像﹐指出该影像的行数,列数,波段数,Datatype, Interleave,空间分辨率,成像时太阳高度角,太阳方位角,日地距离和云量﹔
		
		
	
		\item 使用ENVI指出该影像\textit{中心点像素}的位置;说明查看某像素各波段DN值的方法;
		
		\item 打开\textit{Landsat8}遥感影像的\textit{全色波段},\textit{卷云波段},\textit{图像质量波段},并指出\textit{Landsat8}影像总共的波段数目
		
		
			
		
		
		
		
	\end{enumerate}
	
	\section{实验步骤}
	
	\subsection{在ENVI打开\textit{安徽省宣城市}遥感影像}
	
	
	
	
	
	(1)在\textbf{系统搜索框}中输入\textit{ENVI},打开ENVI(64bit)
	\begin{figure}[H]
		\centering
		\includegraphics[width=.6\textwidth]{open_sea}
		\caption{系统搜索框}
		\label{fig:open1}
	\end{figure}
	\begin{figure}[H]
		\centering
		\includegraphics[width=.6\textwidth]{open_envi}
		\caption{ENVI 5.3 界面}
		\label{fig:open2}
	\end{figure}
	
	(2)在ENVI左上角菜单栏中点击\code{File},选择\code{Open}
	
	\begin{figure}[H]
		\centering
		\includegraphics[width=.6\textwidth]{open_file}
		\caption{打开遥感数据文件}
		\label{fig:open3}
	\end{figure}
	
	(3)点击\verb|Open|之后,弹出了选择文件页面,这时候进入上个实验解压的遥感数据文件夹,选择\code{LC08\_L1TP\_120039\_20191031\_20191114\_01\_T1\_MTL.txt},下图可以看到打开的遥感图像。
		\begin{figure}[H]
		\centering
		\includegraphics[width=.6\textwidth]{open_file2}
		\caption{打开MTL文件}
		\label{fig:open4}
	\end{figure}
	
	\begin{figure}[H]
		\centering
		\includegraphics[width=.6\textwidth]{open_fig}
		\caption{遥感图像}
		\label{fig:open5}
	\end{figure}
	
	
	
	\subsection{使用Data Manager获取文件信息}
	
	(1)在ENVI菜单栏下选择\code{Data Manager}图标,点击打开
		\begin{figure}[H]
		\centering
		\includegraphics[width=.6\textwidth]{open_data}
		\caption{选择Data Manager}
		\label{fig:open6}
	\end{figure}


\begin{figure}[H]
	\centering
	\begin{minipage}[t]{0.48\textwidth}
		\centering
		\includegraphics[height=6.0cm]{open_data2}
	
		\label{fig:open7}
		
	\end{minipage}
	\begin{minipage}[t]{0.48\textwidth}
		\centering 
		\includegraphics[height=5.5cm]{open_data3}
	
		\label{fig:open8}
	\end{minipage}
	\caption{Data Manager界面}
\end{figure}
	(2)在菜单下选择\code{File Information}图标,在这里可以查看遥感文件的行数、列数、波段数、数据存储类型(Datatype)、 存储格式(Interleave)和空间分辨率(Pixel)。
	
	\begin{figure}[H]
		\centering
		\includegraphics[width=.5\textwidth]{open_info}
		\caption{Data Manager的数据信息}
		\label{fig:open9}
	\end{figure}
	

	\begin{table}[H]
		
		
		\centering
		\caption{Data Manager显示的数据}
		\label{tb_1}
		
		\begin{tabular}{ccccccc}
			\toprule[1.5pt]
			
		& 行数   & 列数   & 波段数 & 数据存储类型/Datatype & 存储格式/Interleave & 空间分辨率/Pixel \\ \midrule[0.75pt]
		数据 & 7591 & 7741 & 7   & UInt            & BSQ             & 30 Meters   \\ \bottomrule[1.5pt]
		\end{tabular}
	\end{table}

\subsection{使用View Metadata获取文件信息}

	(1)在ENVI的左侧文件栏中选中遥感数据,并右键,选择\code{View Metadata}
	

	\begin{figure}[H]
	\centering
	\includegraphics[width=.5\textwidth]{open_meta}
	\caption{选中View Metadata}
	\label{fig:open10}
\end{figure}
\begin{figure}[H]
	\centering
	\includegraphics[width=.7\textwidth]{open_meta2}
	\caption{View Metadata界面}
	\label{fig:open11}
\end{figure}


(2)在\code{View Metadata}界面中,单击左侧\code{Imagine Parameters},获取云量(Cloud Cover)、太阳方位角(Sun Azimuth)、太阳高度角(Sun Elevation)、日地距离(Earth Sun Distance)的数据。


\begin{figure}[H]
	\centering
	\includegraphics[width=.7\textwidth]{open_meta3}
	\caption{Imagine Parameters界面}
\end{figure}


	\begin{table}[H]
	
	
	\centering
	\caption{Imagine Parameters显示的数据}
	
	
	\begin{tabular}{ccccc}
		\toprule[1.5pt]
		
	& Cloud Cover & Sun Azimuth             & Sun Elevation          & Earth Sun Distance \\
	 \midrule[0.75pt]
		Data & 0.010000\%         & 156.137001\degree & 42.484127\degree & 0.992966 AU  \\ \bottomrule[1.5pt]
	\end{tabular}
\end{table}

\subsection{中心像素点的确定以及各波段DN值的查看}
(1)已知行数$R$为7591,列数$C$为7741,因此我们确定中心像素点位置应为$(\frac{R}{2},\frac{C}{2})$,即(3795.5,3870.5)

(2)点击左上角红色的标识\code{Cursor Value},可以进行鼠标的取值。

\begin{figure}[H]
	\centering
	\includegraphics[width=.7\textwidth]{open_cu}
	\caption{选择Cursor Value}
\end{figure}
(2)在右上角像素定位框中可以输入图像的像素位置,使得Cursor Value的定位点可以定位到该像素位置上,下面我们分别移动定位到像素点(1,1)和中心像素点(3795.5,3870.5)。
	
	
			\begin{figure}[H]
			\centering
			\begin{minipage}[t]{0.48\textwidth}
				\centering
				\includegraphics[height=6.0cm]{open_cu1p}
				
				
				
			\end{minipage}
			\begin{minipage}[t]{0.48\textwidth}
				\centering 
				\includegraphics[height=6.0cm]{open_cucp}
				
				
			\end{minipage}
			\caption{定位过程}
		\end{figure}
	

	
	\begin{figure}[H]
		\centering
		\begin{minipage}[t]{0.48\textwidth}
			\centering
			\includegraphics[height=6.0cm]{open_cuv1}
				\caption{(1,1)像素点的DN值数据}
		\end{minipage}
		\begin{minipage}[t]{0.48\textwidth}
			\centering 
			\includegraphics[height=6.0cm]{open_cuv2}
				\caption{中心像素点的DN值数据}
			
		\end{minipage}
	
	\end{figure}
	(3)从上图可以看出,(1,1)坐标没有DN值数据,中心像素点各个波段的DN值为[6298, 6938, 7865].对左侧文件栏点击遥感文件,我们可以查看该遥感图像的RGB显示方式,从下图我们可以看出,RGB显示方式为红光波段、绿光波段和蓝光波段,因此中心像素点的DN值准确来说是红光波段DN值6298、绿光波段DN值为6938、 蓝光波段DN值为7865。
	\begin{figure}[H]
		\centering
		\includegraphics[width=.5\textwidth]{open_RGB}
		\caption{查看RGB显示方式}
	\end{figure}
\subsection{打开其他波段}

(1)打开\code{Data Manager},在下图标红界面可以查看所有的波段,我们选中全色波段(Panchromatic)

	\begin{figure}[H]
	\centering
	\begin{minipage}[t]{0.48\textwidth}
	\centering
	\includegraphics[height=6.0cm]{open_all}
		\caption{Load Data}
\end{minipage}
\begin{minipage}[t]{0.48\textwidth}
	\centering 
	\includegraphics[height=6.0cm]{open_pan}
	\caption{选中Panchromatic波段}
	
\end{minipage}
	\end{figure}



(2)单击全色波段(Panchromatic),在下点击\code{Load Data}
	
		\begin{figure}[H]
		\centering
	\begin{minipage}[t]{0.48\textwidth}
	\centering
	\includegraphics[height=6.0cm]{open_pan2}
	
	\caption{Load Data}
	
\end{minipage}
\begin{minipage}[t]{0.48\textwidth}
	\centering 
	\includegraphics[height=6.0cm]{open_pan3}
	\caption{Panchromatic波段图像}
	
\end{minipage}
\end{figure}




(3)回到\code{Data Manager},单击卷云波段(Cirrus),在下点击\code{Load Data}

	\begin{figure}[H]
	\centering
\begin{minipage}[t]{0.48\textwidth}
	\centering
	\includegraphics[height=6.0cm]{open_cir}
	
	\caption{Load Data}
	
\end{minipage}
\begin{minipage}[t]{0.48\textwidth}
	\centering 
	\includegraphics[height=6.0cm]{open_cir2}
	\caption{Cirrus波段图像}
	
\end{minipage}
\end{figure}


(4)回到\code{Data Manager},单击图像质量波段(Quality),在下点击\code{Load Data}

	\begin{figure}[H]
	\centering
\begin{minipage}[t]{0.48\textwidth}
	\centering
	\includegraphics[height=6.0cm]{open_qua1}
	
	\caption{Load Data}
	
\end{minipage}
\begin{minipage}[t]{0.48\textwidth}
	\centering 
	\includegraphics[height=6.0cm]{open_qua2}
	\caption{Quality波段图像}
	
\end{minipage}
\end{figure}




(4)再回到\code{Data Manager},我们可以看出Landsat8总共的波段有12个波段,分别为Coastal aerosol,Blue,Green ,Red ,
Near Infrared ,SWIR 1 ,SWIR 2 ,Panchromatic,Cirrus,Thermal Infrared 1,Thermal Infrared 2和Quality

\begin{figure}[H]
	\centering
	\includegraphics[width=.7\textwidth]{2all}
	\caption{所有波段}
\end{figure}

	\end{document}