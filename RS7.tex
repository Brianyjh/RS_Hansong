\documentclass[12pt,a4paper]{article}
\usepackage{ctex}
%\usepackage{geometry}
%\bibliographystyle{plain}
\usepackage[super]{gbt7714}
%\geometry{a4paper}
\usepackage[margin=2.3cm]{geometry}
\usepackage{graphicx}
\usepackage{float}%强制左对齐1
\usepackage{longtable}
\usepackage{listings} 
\usepackage{framed}
\usepackage{graphicx}
\graphicspath{{figures/}}
%\usepackage{biblatex}
\usepackage{amsmath}
\usepackage{amsfonts}
\usepackage{xcolor}
\usepackage{soul}
\usepackage{color}
\usepackage{booktabs}

\title{\heiti 实验七  \quad ENVI图像分类}
\author{}

\date{ }


\newcommand{\upcite}[1]{\textsuperscript{\textsuperscript{\cite{#1}}}} 
\def\degree{${}^{\circ}$}
\definecolor{Seashell}{RGB}{255, 245, 238} %背景色浅一点的
\definecolor{Firebrick4}{RGB}{255, 0, 0}%文字颜色红一点的
\definecolor{shadecolor}{rgb}{0.92,0.92,0.92}%段落背景
\newcommand{\code}[1]{
	\begingroup
	\sethlcolor{Seashell}%背景色
	\textcolor{Firebrick4}{\hl{#1}}%textcolor里面对应文字颜色
	\endgroup
}

\begin{document}
	
	\maketitle	
	\setcounter{page}{52}
	
	\leftline{\textbf{实验时间}:2021年12月10日}	
	\leftline{\textbf{实验环境}:电脑:Windows 10(64 bit); \quad 软件: ENVI 5.3 \quad SP1(64 bit)}	
	
	\section{实验要求}
	\begin{enumerate}
		
		\item 运用融合后(实验五)的图像对家乡的一幅Landsat8遥感影像,目视确定地物的类别,运用监督分类法进行图像的分类
		
		\item 了解基本的机器学习分类处理方法,区分训练样本与验证样本
		
		
		\item 掌握图像的分类、总分类精度、Kappa系数的计算
		
		
	
		
		
		
	\end{enumerate}
	
	\section{实验步骤}
	
	\subsection{ROI的制作}
	(1)打开一幅上次实验处理后融合的宣城市的Landsat8图像。
	\begin{figure}[H]
		\centering
		\includegraphics[width=.8\textwidth]{cla_01}
	\end{figure}



	(2)打开\code{Region of Interest Tool}ROI处理工具。
\begin{figure}[H]
	\centering
	\includegraphics[width=5cm]{cla_02}
\end{figure}
(3)在左上角添加新建新的ROI,并命名为\code{City},设置颜色为黄色
	\begin{figure}[H]
		\centering
		\begin{minipage}[t]{0.48\textwidth}
			\centering
			\includegraphics[height=5cm]{cla_03}
			\caption{新建ROI}
		\end{minipage}
		\begin{minipage}[t]{0.48\textwidth}
			\centering 
			\includegraphics[height=5cm]{cla_04}
			\caption{城市ROI}
		\end{minipage}
	\end{figure}
	(4)同理设置\code{Land,Forrest,Agricultrue,Water}这些ROI。
	\begin{figure}[H]
		\centering
		\includegraphics[height=5cm]{cla_05}
	\end{figure}

	(3)在地图上看水体,用ROI工具进行选取

	\begin{figure}[H]
	\centering
	\includegraphics[width=5cm]{cla_06}
\end{figure}

(4)选取田地的ROI,右键图像文件,选择\code{Change RGB Bands...
},将RGB波段选为(6,5,2),此时田地显示为嫩绿色。


\begin{figure}[H]
	\centering
	\begin{minipage}[t]{0.48\textwidth}
		\centering
		\includegraphics[height=5cm]{cla_agr}
		\caption{改变RGB波段}
	\end{minipage}
	\begin{minipage}[t]{0.48\textwidth}
		\centering 
		\includegraphics[height=5cm]{cla_agr2}
		\caption{选取波段}
	\end{minipage}
\end{figure}

\begin{figure}[H]
	\centering
	\includegraphics[width=.8\textwidth]{cla_agr3}
\end{figure}
(5)同理选取森林的ROI,右键图像文件,将RGB波段选为(5,4,3),此时森林显示为红色。
\begin{figure}[H]
	\centering
	\begin{minipage}[t]{0.48\textwidth}
		\centering
		\includegraphics[height=5cm]{cla_veg}
		\caption{改变RGB波段}
	\end{minipage}
	\begin{minipage}[t]{0.48\textwidth}
		\centering 
		\includegraphics[height=5cm]{cla_veg2}
		\caption{选取波段}
	\end{minipage}
\end{figure}
(6) 选取城市的RGB波段选为(7,6,4),城市显示为紫黑色
\begin{figure}[H]
	\centering
	\includegraphics[width=.8\textwidth]{cla_city}
\end{figure}
(7)为了保证分类的精准度,我们将每类别的ROI都选择50以上的数量,作为\textbf{训练集},同时制作另外30个作为\textbf{验证集}。


\subsection{ROI的制作分离度评价}

(1)在ROI工具中,点击\code{Options},选择\code{Compute ROl Separability...},选择所有的ROI。

\begin{figure}[H]
	\centering
	\begin{minipage}[t]{0.48\textwidth}
		\centering
		\includegraphics[height=5cm]{cla_compute}
		\caption{Compute ROl Separability}
	\end{minipage}
	\begin{minipage}[t]{0.48\textwidth}
		\centering 
		\includegraphics[height=5cm]{cla_compute2}
		\caption{选取ROI}
	\end{minipage}
\end{figure}
(2)得到\code{ROI Separability Report},我们可以看到各个ROI相互间的Separability接近于2,说明分离度很好。
\begin{figure}[H]
	\centering
	\includegraphics[width=8cm]{cla_compute5}
\end{figure}
\subsection{使用最大似然分类(Maximum Likelihood Classification)分类地物}

(1)在ENVI软件界面的右边\code{Toolbox}中点击\code{classification
},然后选择\code{Supervised classification
},进入\code{Maximum Likelihood Classification}界面,选择大气融合的数据。
\begin{figure}[H]
	\centering
	\begin{minipage}[t]{0.48\textwidth}
		\centering
		\includegraphics[height=5cm]{cla_mlm}
	
	\end{minipage}
	\begin{minipage}[t]{0.48\textwidth}
		\centering 
		\includegraphics[height=5cm]{cla_mlm2}

	\end{minipage}
\end{figure}



(2)由于分类器可能会将背景值一起加入分类(如下),因此在选择数据这个环节,我们还需要加入掩膜(提前制作好):


\begin{figure}[H]
	\centering
	\includegraphics[width=5cm]{cla_mlmout}
	\caption{背景被错误分类的情况}
\end{figure}

\begin{figure}[H]
	\centering
	\begin{minipage}[t]{0.48\textwidth}
		\centering
		\includegraphics[height=5cm]{cla_maskmlm}
		
	\end{minipage}
	\begin{minipage}[t]{0.48\textwidth}
		\centering 
		\includegraphics[height=5cm]{cla_maskmlm2}
		
	\end{minipage}
\end{figure}



(3)选择所有的ROI,因为方法是最大似然法,设置\code{Probability Threshold}为\code{None},设置保存路径。

\begin{figure}[H]
	\centering
	\begin{minipage}[t]{0.48\textwidth}
		\centering
		\includegraphics[height=5cm]{cla_mlm3}
	\end{minipage}
	\begin{minipage}[t]{0.48\textwidth}
		\centering 
		\includegraphics[width=5cm]{cla_mlm4}
	\end{minipage}
\end{figure}
(4)得到我们最终的分类图(背景值已经被处理忽略):

%\begin{figure}[H]
%	\centering
%	\includegraphics[width=8cm]{cla_mlm4}
%\end{figure}
\begin{figure}[H]
	\centering
	\includegraphics[width=8cm]{cla_maskmlm4}
\end{figure}




\subsection{使用混淆矩阵(Confusion Matrix)进行分类精度的评价}

(1)\textbf{Kappa系数}是统计学中度量一致性的指标, 值在[-1,1]. 对于评分系统,一致性就是不同打分人平均的一致性; 对于分类问题,一致性就是模型预测结果和实际分类结果是否一致. kappa系数的计算是基于混淆矩阵, 取值为-1到1之间, 通常大于0\upcite{CSDN}。kappa系数的数学表达如下:
$$K=\frac{P_0-P_e}{1-P_e}$$

(2)在ENVI软件界面的右边\code{Toolbox}中点击\code{classification
},然后选择\code{Post Classification},进入\code{Confusion Matrix using Ground Truth ROIs}界面,选择经过分类的数据文件。

\begin{figure}[H]
	\centering
	\begin{minipage}[t]{0.48\textwidth}
		\centering
		\includegraphics[height=5cm]{cla_con}
	\end{minipage}
	\begin{minipage}[t]{0.48\textwidth}
		\centering 
		\includegraphics[width=5cm]{cla_con2}
	\end{minipage}
\end{figure}
(3)因为我们事前准备了验证集(每类地物另外选择了30多个ROI),软件自动帮我们选匹配好了,于是继续点击\code{OK}。

\begin{figure}[H]
	\centering
	\begin{minipage}[t]{0.48\textwidth}
		\centering
		\includegraphics[height=5cm]{cla_con3}
	\end{minipage}
	\begin{minipage}[t]{0.48\textwidth}
		\centering 
		\includegraphics[width=6cm]{cla_con4}
	\end{minipage}
\end{figure}
(4)最终得到了我们的混淆矩阵数据:

\begin{figure}[H]
	\centering
	\includegraphics[width=.8\textwidth]{cla_con5}
\end{figure}



在这里,我们可以看到总分类精度(Overall Accuracy)为$99.3523\%$,Kappa系数为$0.9866$,说明分类精度很高。



\section{参考文献}

	\bibliography{RS}



\section{遥感实验的心得与总结}




这学期很幸运能够接触到\textbf{遥感}这门课程,本来我经常打数模,对计算机类的东西有很多接触,暑假也不停地在学\textbf{机器学习},也遇过需要处理\textbf{地理信息}相关的数学建模题目,但是没有更深一步地学习地理信息的处理,学会了遥感这门课,我可以提高自己的信息素养能力,也是多多掌握了一个软件的操作,也是找到一把打开新世界的钥匙。

软件的使用其实都大同小异,很多东西掌握一个就存在着快速掌握其他软件的可能,有了这学期ENVI的学习,肯定能让我后续学习GIS软件打下基础。老师也说过,软件的操作、鼠标的移动点击,其实不是最重要的,而对于遥感这门课的学习,知识体系的构建与专业技能的掌握与积累,才是学习遥感实验这门课这个过程中最最重要的东西。

在处理各个遥感文件的时候,我还是能体会到大数据时代的鲜明特征,当遥感实验的\code{Work project}从1G涨到了16G,巨额数据的处理仍需要算法和软件的熟练掌握,所以,在这个时候,多多学习数据处理的知识,面对大数据时代,要不断与时俱进。






\end{document}