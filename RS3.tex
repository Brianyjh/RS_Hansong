\documentclass[12pt,a4paper]{article}
\usepackage{ctex}
%\usepackage{geometry}
%\bibliographystyle{plain}
\usepackage[super]{gbt7714}
%\geometry{a4paper}
\usepackage[margin=2.3cm]{geometry}
\usepackage{graphicx}
\usepackage{float}%强制左对齐1
\usepackage{longtable}
\usepackage{listings} 

\usepackage{graphicx}
\graphicspath{{figures/}}
%\usepackage{biblatex}
\usepackage{amsmath}
\usepackage{amsfonts}
\usepackage{xcolor}
\usepackage{soul}
\usepackage{booktabs}
\title{\heiti 实验三 \quad ENVI影像裁剪与镶嵌}
\author{}

\date{ }


\newcommand{\upcite}[1]{\textsuperscript{\textsuperscript{\cite{#1}}}} 
\def\degree{${}^{\circ}$}
\definecolor{Seashell}{RGB}{255, 245, 238} %背景色浅一点的
\definecolor{Firebrick4}{RGB}{255, 0, 0}%文字颜色红一点的
\newcommand{\code}[1]{
	\begingroup
	\sethlcolor{Seashell}%背景色
	\textcolor{Firebrick4}{\hl{#1}}%textcolor里面对应文字颜色
	\endgroup
}

\begin{document}
	
	\maketitle	
\setcounter{page}{14}

\leftline{\textbf{实验时间}:2021年11月30日}	
\leftline{\textbf{实验环境}:电脑:Windows 10(64 bit); \quad 软件: ENVI 5.3 \quad  SP1(64bit)}	
	
	\section{实验要求}
		\begin{enumerate}
		
	\item 下载多幅家乡的Landsat8遥感影像﹐使用裁剪与镶嵌方法将图像进行合并。
	\item 根据合并的图像,制作家乡的无云且色彩一致的影像(作为制图底图)。
	\item 掌握裁剪、镶嵌、融合的基本ENVI操作

	\end{enumerate}
	
	\section{实验步骤}
	
	\subsection{下载多副宣城市Landsat8影像,覆盖宣城全域}
	(1)在\textbf{地理空间数据云门户网站}(\underline{http://www.gscloud.cn})中下载多副宣城市影像,并解压后用ENVI打开。
	
	\begin{figure}[H]
		\centering
		\begin{minipage}[t]{0.48\textwidth}
			\centering
			\includegraphics[height=5.0cm]{mer_01}	
			\caption{选择多幅影像}
		\end{minipage}
		\begin{minipage}[t]{0.48\textwidth}
			\centering 
			\includegraphics[height=5.0cm]{mer_012}
				\caption{打开多幅影像}
		\end{minipage}
	\end{figure}

	\subsection{基于地理坐标的影像镶嵌}
	(1)在ENVI软件界面的右边\code{Toolbox}中点击\code{Mosaicking},然后选择\code{Seamless Mosaic},进入\code{Seamless Mosaic}界面。
	
	\begin{figure}[H]
	\centering
	\begin{minipage}[t]{0.48\textwidth}
		\centering
		\includegraphics[height=7cm]{mer_sm}
		\caption{选择	Seamless Mosaic}
	\end{minipage}
	\begin{minipage}[t]{0.48\textwidth}
		\centering 
		\includegraphics[height=7cm]{mer_sm1}
		\caption{Seamless Mosaic界面}
	\end{minipage}
\end{figure}

	(2)选择两幅多光谱(MultiSpectral)影像
	\begin{figure}[H]
	\centering
	\begin{minipage}[t]{0.48\textwidth}
		\centering
		\includegraphics[height=7cm]{mer_2fig}
		\caption{选择两幅多光谱(MultiSpectral)影像}
	\end{minipage}
	\begin{minipage}[t]{0.48\textwidth}
		\centering 
		\includegraphics[height=7cm]{mer_2fig2}
		\caption{选择后的 Seamless Mosaic界面}
	\end{minipage}
\end{figure}
	(3)点击\code{Order}调整图像顺序
		\begin{figure}[H]
		\centering
		\includegraphics[width=.6\textwidth]{mer_ord}
	\end{figure}
(4)在\code{Seamless Mosaic}主界面数据栏中,一共有Scene  Correction、Data Ignore Value
Color Matching Action和Feathering Distance (Pixels)这几类数据信息,我们修改羽化距离(Feathering Distance (Pixels))为10 m。

\begin{figure}[H]
	\centering
	\includegraphics[width=.5\textwidth]{mer_sm2}
		\caption{修改羽化距离}
\end{figure}


(4)点击\code{Auto Generate Seamling},自动生成拼接线。
 
	\begin{figure}[H]
	\centering
	\begin{minipage}[t]{0.48\textwidth}
		\centering
		\includegraphics[height=7cm]{mer_auto}
		\caption{点击Auto Generate Seamling}
	\end{minipage}
	\begin{minipage}[t]{0.48\textwidth}
		\centering 
		\includegraphics[height=7cm]{mer_auto2}
		\caption{Auto Generate Seamling}
	\end{minipage}
\end{figure}
\begin{figure}[H]
	\centering
	\includegraphics[width=.5\textwidth]{mer_auto3}
	\caption{自动生成的拼接线}
\end{figure}
(5)点击\code{Color Correction
}进行颜色校正,勾选直方图匹配 \code{Histogram Matching},并选择重叠区\code{Overlap Area Only
}。
\begin{figure}[H]
	\centering
	\includegraphics[width=.5\textwidth]{mer_col}
	\caption{颜色校正}
\end{figure}
(6)点击\code{Seamlines/Feathering}进行羽化设置,勾选\code{Apply Seamlines}应用接边线和\code{Seamline Feathering}进行接边线羽化。
\begin{figure}[H]
	\centering
	\includegraphics[width=.5\textwidth]{mer_fea}
	\caption{设置羽化}
\end{figure}
(8)点击\code{Show Preview}可以预览处理后的图像
\begin{figure}[H]
	\centering
	\includegraphics[width=.5\textwidth]{mer_pre}
	\caption{预览图像}
\end{figure}
(9)点击\code{Export},设置好输出文件夹,并且设置\code{Resampling Method}重采样方法为\code{Bilinear}双线性算法。最后导出图像。
\begin{figure}[H]
	\centering
	\includegraphics[height=5.5cm]{mer_out}
	\caption{导出}
\end{figure}
\begin{figure}[H]
	\centering
	\includegraphics[width=.6\textwidth]{mer_smend}
	\caption{Seamless Mosaic最终处理后的图像}
\end{figure}

	\subsection{基于像素的影像镶嵌}
	
	(1)回到两幅影像的界面
	
	\begin{figure}[H]
		\centering
		\includegraphics[width=.6\textwidth]{mer_ori}
		
	\end{figure}
	
	(2)在ENVI软件界面的右边\code{Toolbox}中点击\code{Mosaicking},然后选择\code{Pixel Based Mosaic},进入\code{Pixel Based Mosaic}界面。
			\begin{figure}[H]
			\centering
			\begin{minipage}[t]{0.48\textwidth}
				\centering
				\includegraphics[height=6cm]{mer_pbm}
				\caption{选择	Pixel Based Mosaic}
			\end{minipage}
			\begin{minipage}[t]{0.48\textwidth}
				\centering 
				\includegraphics[height=6cm]{mer_pbm2}
				\caption{Pixel Based Mosaic界面}
			\end{minipage}
		\end{figure}
		(3)在\code{Pixel Based Mosaic}界面选择\code{lmport Files...
		},选中两个图像的多光谱图像,点击\code{OK}
	\begin{figure}[H]
		\centering
		\begin{minipage}[t]{0.48\textwidth}
			\centering
			\includegraphics[height=6cm]{mer_imp}
			\caption{lmport Files}
		\end{minipage}
		\begin{minipage}[t]{0.48\textwidth}
			\centering 
			\includegraphics[height=6cm]{mer_imp2}
			\caption{选中多光谱图像}
		\end{minipage}
	\end{figure}

	
		(4)设置图片尺寸,这里设置尺寸为\code{15000 ×15000}。
		(尺寸大小不得小于任一图像)

		\begin{figure}[H]
		\centering
		\begin{minipage}[t]{0.48\textwidth}
			\centering
			\includegraphics[height=5cm]{mer_size}
			\caption{设置尺寸}
		\end{minipage}
		\begin{minipage}[t]{0.48\textwidth}
			\centering 
			\includegraphics[height=6cm]{mer_entry00}
			\caption{Pixel Mosaic}
		\end{minipage}
	\end{figure}








	(5)在\code{Pixel Mosaic}界面移动图像,对着上面的图像右键,点击\code{Edit Entry}。设置\code{Feathering distance}为\code{10},\code{Linear Stretch}为\code{0.00\%},完成后对下面的图像采取相同的操作。
\begin{figure}[H]
	\centering
	\begin{minipage}[t]{0.48\textwidth}
		\centering
		\includegraphics[height=5cm]{mer_right}
		\caption{右键点击}
	\end{minipage}
	\begin{minipage}[t]{0.48\textwidth}
		\centering 
		\includegraphics[height=5cm]{mer_entry}
		\caption{Entry界面}
	\end{minipage}
\end{figure}


特别地,在这里我们发现在进行\code{Pixel Mosaic}的时候,发现图像的黑边影响着镶嵌,这里我们参考网络上的博客(\underline{https://blog.csdn.net/chenpx1224/article/details/19298311}),选择顶层的图像,右键点击选择\code{Edit Entry},将\code{Background See Through}中\code{Date Value to lgnore}值设置为\code{0}。
\begin{figure}[H]
	\centering
	\begin{minipage}[t]{0.48\textwidth}
		\centering
		\includegraphics[height=5cm]{other2}
		\caption{黑边}
	\end{minipage}
	\begin{minipage}[t]{0.48\textwidth}
		\centering 
		\includegraphics[height=5cm]{other}
		\caption{忽略背景}
	\end{minipage}
\end{figure}


(6)设置完以上操作后,在\code{Pixel Mosaic}界面点击\code{File},点击\code{Apply},设置保存路径
\begin{figure}[H]
	\centering
	\begin{minipage}[t]{0.48\textwidth}
		\centering
		\includegraphics[height=6cm]{mer_apl}
		\caption{Apply}
	\end{minipage}
	\begin{minipage}[t]{0.48\textwidth}
		\centering 
		\includegraphics[width=5cm]{mer_load}
		\caption{保存图像}
	\end{minipage}
\end{figure}
(6)得到镶嵌后的图像如下:
	\begin{figure}[H]
	\centering
	\includegraphics[width=.6\textwidth]{mer_pbmout}
\end{figure}
(7)我们可以看到图像背景都是黑的,这时候使用一个小技巧,把黑色部分都去掉。在右边\code{Toolbox}中选择\code{Raster Management},选择其中的\code{Edit ENVI Header},并选择镶嵌后的图像。
\begin{figure}[H]
	\centering
	\begin{minipage}[t]{0.48\textwidth}
		\centering
		\includegraphics[height=6cm]{mer_rm}
		\caption{Edit ENVI Header}
	\end{minipage}
	\begin{minipage}[t]{0.48\textwidth}
		\centering 
		\includegraphics[width=5cm]{mer_fs}
		\caption{选择图像}
	\end{minipage}
\end{figure}

(8)在\code{Set Raster Metadata
}界面点击左上角的加号,点选\code{Data Ignore Value},再点击\code{OK}。
\begin{figure}[H]
	\centering
	\begin{minipage}[t]{0.48\textwidth}
		\centering
		\includegraphics[height=6cm]{mer_rsadd}
		\caption{点击Add}
	\end{minipage}
	\begin{minipage}[t]{0.48\textwidth}
		\centering 
		\includegraphics[width=5cm]{mer_div}
		\caption{选择Data Ignore Value}
	\end{minipage}
\end{figure}
(8)回到\code{Set Raster Metadata
}界面,在最底部找到\code{Data Ignore Value},设置为\code{0}。再点击\code{OK}
\begin{figure}[H]
	\centering
	\begin{minipage}[t]{0.48\textwidth}
		\centering
		\includegraphics[height=6cm]{mer_div0}
		\caption{点击Add}
	\end{minipage}
	\begin{minipage}[t]{0.48\textwidth}
		\centering 
		\includegraphics[width=5cm]{mer_div}
		\caption{选择Data Ignore Value}
	\end{minipage}
\end{figure}

(9)去除黑边后,我们得到我们的图像如下:
	\begin{figure}[H]
	\centering
	\includegraphics[width=.6\textwidth]{mer_end}
	\caption{Pixel Mosaic最终处理后的图像}
\end{figure}

\subsection{利用Regions of Interest对遥感图像进行裁剪}


(1)我们从超星学习通下载\code{CityRegion.rar}获取全国的shp文件,并解压。
\begin{figure}[H]
	\centering
	\begin{minipage}[t]{0.48\textwidth}
		\centering
		\includegraphics[height=4cm]{cut_1}
		\caption{下载shp文件}
	\end{minipage}
	\begin{minipage}[t]{0.48\textwidth}
		\centering 
		\includegraphics[width=5cm]{cut_2}
		\caption{解压后的文件}
	\end{minipage}
\end{figure}

(2)打开ArcMap10.2,在软件界面,选择“添加数据”,选择我们下载的\code{City\_region.shp}
\begin{figure}[H]
	\centering
	\begin{minipage}[t]{0.48\textwidth}
		\centering
		\includegraphics[height=5cm]{cut_3}
		\caption{打开ArcMap}
	\end{minipage}
	\begin{minipage}[t]{0.48\textwidth}
		\centering 
		\includegraphics[height=5cm]{cut_4}
		\caption{添加数据}
	\end{minipage}
\end{figure}
(3)打开shp文件后,我们点击“选择要素”的工具
\begin{figure}[H]
	\centering
	\begin{minipage}[t]{0.48\textwidth}
		\centering
		\includegraphics[height=5cm]{cut_5}
		\caption{打开shp文件}
	\end{minipage}
	\begin{minipage}[t]{0.48\textwidth}
		\centering 
		\includegraphics[height=5cm]{cut_6}
		\caption{选择要素}
	\end{minipage}
\end{figure}
(4)点击宣城的范围,在左侧“内容列表”中,右键点击shp数据,点击“数据”,再点击“导出数据”
\begin{figure}[H]
	\centering
	\begin{minipage}[t]{0.48\textwidth}
		\centering
		\includegraphics[height=5cm]{cut_7}
		\caption{导出数据}
	\end{minipage}
	\begin{minipage}[t]{0.48\textwidth}
		\centering 
		\includegraphics[height=5cm]{cut_8}
		\caption{导出时进行的设置}
	\end{minipage}
\end{figure}
(5)设置导出的数据为“所选要素”,并设置输出路径,于是得到了我们宣城市的shp文件。

	\begin{figure}[H]
	\centering
	\includegraphics[width=.4\textwidth]{cut_9}
\end{figure}

(6)在ENVI打开宣城市的shp文件
	\begin{figure}[H]
	\centering
	\includegraphics[width=.4\textwidth]{cut_shp}
\end{figure}

(7)在右侧\code{Toolbox}中选择\code{Regions of Interest}下的\code{Vector to ROI
},选择宣城市的shp,点击\code{OK}。
\begin{figure}[H]
	\centering
	\begin{minipage}[t]{0.48\textwidth}
		\centering
		\includegraphics[height=5cm]{cut_v2r}
		\caption{选择Regions of Interest}
	\end{minipage}
	\begin{minipage}[t]{0.48\textwidth}
		\centering 
		\includegraphics[height=5cm]{cut_v2r2}
		\caption{选择shp文件}
	\end{minipage}
\end{figure}
(8)在\code{Convert Vector to ROI}选择\code{All records to a single ROI
},选择之前镶嵌的图像作为底图。
\begin{figure}[H]
	\centering
	\begin{minipage}[t]{0.48\textwidth}
		\centering
		\includegraphics[height=5cm]{cut_v2rall}
		\caption{Convert Vector to ROI界面}
	\end{minipage}
	\begin{minipage}[t]{0.48\textwidth}
		\centering 
		\includegraphics[width=6cm]{cut_smv2r}
		\caption{选择底图}
	\end{minipage}
\end{figure}
(9)在右侧\code{Toolbox}中选择\code{Regions of Interest}下的\code{Subset Data from ROIs},选择宣城市的shp,点击\code{OK}。
\begin{figure}[H]
	\centering
	\begin{minipage}[t]{0.48\textwidth}
		\centering
		\includegraphics[height=5cm]{cut_sdfr}
		\caption{选择Subset Data from ROIs}
	\end{minipage}
	\begin{minipage}[t]{0.48\textwidth}
		\centering 
		\includegraphics[height=5cm]{cut_sdfr2}
		\caption{选择底图}
	\end{minipage}
\end{figure}

(10)在\code{Spatial Subset via ROl Parameters}界面,ROI选择\code{EVF:xuancheng.shp},\code{Mask pixels outside of ROI?}设置为\code{Yes},\code{Mask Background value}设置为\code{0},最后设置路径。
\begin{figure}[H]
	\centering
	\begin{minipage}[t]{0.48\textwidth}
		\centering
		\includegraphics[height=5cm]{cut_sdfr3}
		\caption{界面设置}
	\end{minipage}
	\begin{minipage}[t]{0.48\textwidth}
		\centering 
		\includegraphics[width=5cm]{cut_sdfr4}
		\caption{进行裁剪}
	\end{minipage}
\end{figure}

(11)最终得到宣城市无云且色彩一致的影像。

	\begin{figure}[H]
	\centering
	\includegraphics[width=.8\textwidth]{cut_final}
\end{figure}



\end{document}