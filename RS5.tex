\documentclass[12pt,a4paper]{article}
\usepackage{ctex}
%\usepackage{geometry}
%\bibliographystyle{plain}
\usepackage[super]{gbt7714}
%\geometry{a4paper}
\usepackage[margin=2.3cm]{geometry}
\usepackage{graphicx}
\usepackage{float}%强制左对齐1
\usepackage{longtable}
\usepackage{listings} 
\usepackage{framed}
\usepackage{graphicx}
\graphicspath{{figures/}}
%\usepackage{biblatex}
\usepackage{amsmath}
\usepackage{amsfonts}
\usepackage{xcolor}
\usepackage{soul}
\usepackage{color}
\usepackage{booktabs}

\title{\heiti 实验五 \quad ENVI图像融合}


\author{}

\date{ }


\newcommand{\upcite}[1]{\textsuperscript{\textsuperscript{\cite{#1}}}} 
\def\degree{${}^{\circ}$}
\definecolor{Seashell}{RGB}{255, 245, 238} %背景色浅一点的
\definecolor{Firebrick4}{RGB}{255, 0, 0}%文字颜色红一点的
\definecolor{shadecolor}{rgb}{0.92,0.92,0.92}%段落背景
\newcommand{\code}[1]{
	\begingroup
	\sethlcolor{Seashell}%背景色
	\textcolor{Firebrick4}{\hl{#1}}%textcolor里面对应文字颜色
	\endgroup
}

\begin{document}
	
	\maketitle	
	\setcounter{page}{38}
	
	\leftline{\textbf{实验时间}:2021年12月2日}	
	\leftline{\textbf{实验环境}:电脑:Windows 10(64 bit); \quad 软件: ENVI 5.3 \quad SP1(64bit)}	
	
	\section{实验要求}
	\begin{enumerate}
		
		\item 对自己家乡的一幅Landsat8遥感影像进行分别进行预处理,准备进行图像融合。
		
		\item 选择适当的图像融合方法,进行图像融合,获取获取空间分辨率15m的多光谱Landsat8影像
		
		
		\item 掌握图像融合的步骤与方法。
		
		
		
	\end{enumerate}
	
	\section{实验步骤}
	
	\subsection{使用Gram-Schmidt法进行图像融合}
	
	

	(1)对于图像融合方法,我们可以看下表的比较,为了让我们的图像保持高保真,我们可以使用Gram-Schmidt 法进行图像融合。
	
	







\begin{center}
	\textbf{表 1}~~图像融合方法比较\\
	\vspace *{.5\baselineskip}
	\resizebox{\textwidth}{15mm}{
			\begin{tabular}{cc}
			\toprule[1.5pt]
			
			方法 &特点  \\  \midrule[0.75pt]
			HIS变换          & 纹理改善,空间保持较好。光谱信息损失较大大,受波段限制                       \\ 
			Brovey变换       & 光谱信息保持较好,受波段限制。                                   \\
			乘积运算( CN)      & 对大的地貌类型效果好,同时可用于多光谱与高光谱的融合。                       \\
			PCA变换          & 无波段限制,光谱保持好。第一主成分信息高度集中,色调发生较大变化,                 \\
			Gram-schmidt   & 改进了PCA中信息过分集中的问题,不受波段限制,较好的保持空间纹理信息,尤其能高保真保持光谱特征。 \\
			Pan sharpening & 专为最新高空间分辨率影像设计,能较好保持影像的纹理和光谱信息。                   \\ \bottomrule[1.5pt]
		\end{tabular}
		}
\end{center}








	(2)打开一幅宣城市的Landsat8图像与与之对应的进行绝对大气校正的图像。
	\begin{figure}[H]
		\centering
		\includegraphics[width=.6\textwidth]{air}
	\end{figure}
	\begin{figure}[H]
		\centering
		\begin{minipage}[t]{0.48\textwidth}
			\centering
			\includegraphics[height=4.5cm]{air}
			\caption{Landsat8图像}
		\end{minipage}
		\begin{minipage}[t]{0.48\textwidth}
			\centering 
			\includegraphics[height=4.5cm]{merge_01}
			\caption{大气校正图像}
		\end{minipage}
	\end{figure}
	
	
	(3)在右侧\code{Toolbox}中的\code{Image sharpening
	}选择\code{Gram-Schmidt Pan Sharpening}
	\begin{figure}[H]
		\centering
		\includegraphics[height=5cm]{merge_gs}
	\end{figure}
	
	
	(4)选择大气校正的图像作为低分辨率图像,选择全色图像作为高分辨图像
	\begin{figure}[H]
		\centering
		\begin{minipage}[t]{0.48\textwidth}
			\centering
			\includegraphics[height=4cm]{merge_gs2}
			\caption{选择低分辨率图像}
		\end{minipage}
		\begin{minipage}[t]{0.48\textwidth}
			\centering 
			\includegraphics[height=4cm]{merge_gs3}
			\caption{选择高分辨率图像}
		\end{minipage}
	\end{figure}
	(5)进入\code{Pan Sharpening Parameters}页面,设置\code{Sensor}为\code{landsat8\_oli},设置\code{Resampling}重采样方法为\code{Bilinear}双重插值法。最后设置输出路径,点击\code{OK}。
		\begin{figure}[H]
		\centering
		\includegraphics[height=4cm]{merge_gs6}
	\end{figure}
	(6)融合后得到的结果如下,我们发现整个图像融合后变白,我们可以发现背景值都是负值,而图像的背景值为正值,ENVI为了保证整体效果,发生灰度拉开。为了保证图像的视觉效果,我们应该去除负值。
		\begin{figure}[H]
		\centering
		\includegraphics[height=4cm]{merge_gs7}
		\caption{初次融合的结果}
	\end{figure}
	\begin{figure}[H]
		\centering
		\includegraphics[height=4cm]{merge_gs8}
		\caption{背景值为负值}
	\end{figure}

	(7)建立掩膜可以防止背景值也参与融合,而融合的图像的空间分辨率与全色影像是一致的,因此我们选择全色影像进行掩膜的建立,首先我们打开\code{Data Manager}点击全色影像选择\code{Load Grayscale}加载灰度影像,点击\code{Raster Manager}中的\code{Build Mask}。
	
	


	\begin{figure}[H]
	\centering
	\begin{minipage}[t]{0.48\textwidth}
		\centering
	\includegraphics[height=5cm]{merge_grey}
\caption{导入灰度影像}
	\end{minipage}
	\begin{minipage}[t]{0.48\textwidth}
		\centering 
		\includegraphics[height=5cm]{merge_mask}
	\caption{Build Mask}
	\end{minipage}
\end{figure}

		(8)点击\code{Options}点击\code{Import Data Range}选择包含数据的图像,选择全色影像作为数据文件。
	
	\begin{figure}[H]
		\centering
		\begin{minipage}[t]{0.48\textwidth}
			\centering
			\includegraphics[height=5cm]{merge_mask4}
			\caption{Import Data Range}
		\end{minipage}
		\begin{minipage}[t]{0.48\textwidth}
			\centering 
			\includegraphics[height=5cm]{merge_mask5}
			\caption{选择全色影像}
		\end{minipage}
	\end{figure}
	
	(9)在\code{lnput for Data Range Mask}界面选取掩膜数据的范围点击,我们掩膜的是值为0的背景值数据。选定好数据范围后,我们设定文件保存路径,点击\code{OK}。
		\begin{figure}[H]
		\centering
		\begin{minipage}[t]{0.48\textwidth}
			\centering
			\includegraphics[height=5cm]{merge_mask6}
			\caption{设置数据范围}
		\end{minipage}
		\begin{minipage}[t]{0.48\textwidth}
			\centering 
			\includegraphics[height=5cm]{merge_mask7}
			\caption{设置保存路径}
		\end{minipage}
	\end{figure}
		\begin{figure}[H]
		\centering
		\includegraphics[height=5cm]{merge_mask8}
		\caption{开始制作掩膜}
	\end{figure}
	\begin{figure}[H]
		\centering
		\includegraphics[height=5cm]{merge_mask9}
		\caption{生成的掩膜}
	\end{figure}
	
	(10)点击\code{Raster Manager}中的\code{Apply Mask},选择经过融合的图像作为底图。
		\begin{figure}[H]
		\centering
		\begin{minipage}[t]{0.48\textwidth}
			\centering
			\includegraphics[height=5cm]{merge_app2}
			\caption{Apply Mask}
		\end{minipage}
		\begin{minipage}[t]{0.48\textwidth}
			\centering 
			\includegraphics[height=5cm]{merge_mask7}
			\caption{选定底图}
		\end{minipage}
	\end{figure}
	(11)点击\code{Select Mask Band}选择刚制作好的掩膜文件。
	\begin{figure}[H]
		\centering
		\begin{minipage}[t]{0.48\textwidth}
			\centering
			\includegraphics[height=5cm]{merge_app3}
			\caption{Select Mask Band}
		\end{minipage}
		\begin{minipage}[t]{0.48\textwidth}
			\centering 
			\includegraphics[height=5cm]{merge_app4}
			\caption{选定掩膜文件}
		\end{minipage}
	\end{figure}
		(12)设置\code{Mask Value}为\code{0},并设置文件保存路径。
	\begin{figure}[H]
		\centering
		\begin{minipage}[t]{0.48\textwidth}
			\centering
			\includegraphics[height=4cm]{merge_app6}
			\caption{设置掩膜的值}
		\end{minipage}
		\begin{minipage}[t]{0.48\textwidth}
			\centering 
			\includegraphics[width=5cm]{merge_app7}
			\caption{应用掩膜文件}
		\end{minipage}
	\end{figure}
最终得到了我们的掩膜处理的图像。
	\begin{figure}[H]
	\centering
	\includegraphics[height=5cm]{merge_app8}
	\caption{掩膜后的融合图像}
\end{figure}
(13)为了消除背景黑色,我们点击\code{Raster Management}中的\code{Edit ENVI Header},选择刚刚生成的掩膜处理的融合图像。
	\begin{figure}[H]
	\centering
	\begin{minipage}[t]{0.48\textwidth}
		\centering
		\includegraphics[height=5cm]{merge_bla}
		\caption{Edit ENVT Header}
	\end{minipage}
	\begin{minipage}[t]{0.48\textwidth}
		\centering 
		\includegraphics[height=5cm]{merge_bla2}
		\caption{选择融合图像}
	\end{minipage}
\end{figure}
(14)在加载出来的窗口选择左上角的\code{Add},选择\code{Data Ignore Value}。
	\begin{figure}[H]
	\centering
	\begin{minipage}[t]{0.48\textwidth}
		\centering
		\includegraphics[height=5cm]{merge_bla3}
		\caption{点击Add}
	\end{minipage}
	\begin{minipage}[t]{0.48\textwidth}
		\centering 
		\includegraphics[height=5cm]{merge_bla4}
		\caption{选择Data Ignore Value}
	\end{minipage}
\end{figure}
(15)最后设定\code{Data Ignore Value}为\code{0},得到了我们最终的图像.



	\begin{figure}[H]
	\centering
	\begin{minipage}[t]{0.48\textwidth}
		\centering
		\includegraphics[height=5cm]{merge_bla5}
		\caption{设置Data Ignore Value}
	\end{minipage}
	\begin{minipage}[t]{0.48\textwidth}
		\centering 
		\includegraphics[height=5cm]{merge_END}
		\caption{最终的融合图像}
	\end{minipage}
\end{figure}
\end{document}