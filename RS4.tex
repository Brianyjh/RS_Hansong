\documentclass[12pt,a4paper]{article}
\usepackage{ctex}
%\usepackage{geometry}
%\bibliographystyle{plain}
\usepackage[super]{gbt7714}
%\geometry{a4paper}
\usepackage[margin=2.3cm]{geometry}
\usepackage{graphicx}
\usepackage{float}%强制左对齐1
\usepackage{longtable}
\usepackage{listings} 
\usepackage{framed}
\usepackage{graphicx}
\graphicspath{{figures/}}
%\usepackage{biblatex}
\usepackage{amsmath}
\usepackage{amsfonts}
\usepackage{xcolor}
\usepackage{soul}
\usepackage{color}
\usepackage{booktabs}

\title{\heiti 实验四 \quad ENVI大气校正}
\author{}

\date{ }


\newcommand{\upcite}[1]{\textsuperscript{\textsuperscript{\cite{#1}}}} 
\def\degree{${}^{\circ}$}
\definecolor{Seashell}{RGB}{255, 245, 238} %背景色浅一点的
\definecolor{Firebrick4}{RGB}{255, 0, 0}%文字颜色红一点的
\definecolor{shadecolor}{rgb}{0.92,0.92,0.92}%段落背景
\newcommand{\code}[1]{
	\begingroup
	\sethlcolor{Seashell}%背景色
	\textcolor{Firebrick4}{\hl{#1}}%textcolor里面对应文字颜色
	\endgroup
}

\begin{document}
	
	\maketitle	
	\setcounter{page}{28}
	
	\leftline{\textbf{实验时间}:2021年12月2日}	
	\leftline{\textbf{实验环境}:电脑:Windows 10(64 bit); \quad 软件: ENVI 5.3 \quad SP1(64bit)}	
	
	\section{实验要求}
	\begin{enumerate}
		
		\item 对自己家乡的一幅Landsat8遥感影像进行分别进行相对大气校正和绝对大气校正
		
		\item 选择典型地物(包括水,植被,土壤等)比较绝对大气校正前后的反射波谱曲线
		
		
		\item 掌握相对大气校正与绝对大气校正的步骤与方法
		
	\end{enumerate}
	
	\section{实验步骤}
	
	\subsection{相对大气校正}
	(1)打开一幅宣城市的Landsat8图像。
	\begin{figure}[H]
		\centering
		\includegraphics[width=.6\textwidth]{air}
		
	\end{figure}
	
	(2)在右侧\code{Toolbox}中选择\code{Radiometric Correction}中\code{Atmospheric Correction Module}下的\code{Quick Atmospheric Correction (QUAC)},在选择文件界面选择多光谱文件。
	
	\begin{figure}[H]
		\centering
		\begin{minipage}[t]{0.48\textwidth}
			\centering
			\includegraphics[height=5.0cm]{air_QUAC}	
			\caption{选择QUAC}
		\end{minipage}
		\begin{minipage}[t]{0.48\textwidth}
			\centering 
			\includegraphics[height=5.0cm]{air_fs}
			\caption{选择多光谱影像}
		\end{minipage}
	\end{figure}
	
	(3)在\code{QUAC}界面,\code{Sensor Type}选择\code{Landsat TM/ETM/0LT
	},设置好文件输出的路径,点击\code{OK}。
	\begin{figure}[H]
		\centering
		\begin{minipage}[t]{0.48\textwidth}
			\centering
			\includegraphics[height=4.0cm]{air_quout}	
			\caption{选择QUAC}
		\end{minipage}
		\begin{minipage}[t]{0.48\textwidth}
			\centering 
			\includegraphics[width=5.0cm]{air_outqu}
			\caption{选择多光谱影像}
		\end{minipage}
	\end{figure}
	(4)最终得到了我们的快速大气校正的图像。
	\begin{figure}[H]
		\centering
		\includegraphics[width=.6\textwidth]{air2}
		
	\end{figure}
	
	\subsection{辐射定标}
	
	(1)重新打开一幅宣城市的影像
	\begin{figure}[H]
		\centering
		\includegraphics[width=.6\textwidth]{air}
	\end{figure}
	(2)在右侧\code{Toolbox}中选择\code{Radiometric Correction}下的\code{Radiometric Correction},在选择文件界面选择多光谱文件。在\code{Radiometric Calibration
	}设置界面中,设置\code{Output Interleave}(文件输出格式)为\code{BIL},点击\code{Apply FLAASH Settings},使得\code{Scale Factor}(缩放系数)为\code{0.10},设置好文件的输出文件夹,点击\code{OK}。
	\begin{figure}[H]
		\centering
		\begin{minipage}[t]{0.48\textwidth}
			\centering
			\includegraphics[height=5.0cm]{air_fcc}	
			\caption{选择Radiometric Correction}
		\end{minipage}
		\begin{minipage}[t]{0.48\textwidth}
			\centering 
			\includegraphics[height=5.0cm]{air_setrc}
			\caption{设置各种参数}
		\end{minipage}
	\end{figure}
	(3)最终得到我们的辐射定标图像。
	\begin{figure}[H]
		\centering
		\includegraphics[width=.6\textwidth]{air_rcout}
	\end{figure}
	\subsection{绝对大气校正与图像变化对比}
	\subsubsection{绝对大气校正}
	(1)在右侧\code{Toolbox}中选择\code{Radiometric Correction}中\code{Atmospheric Correction Module}下的\code{FLAASH Atmospheric Correction},在选择文件界面选择多光谱文件。我们点击\code{Input Radiance Image}进行图像文件的导入。
	\begin{figure}[H]
		\centering
		\begin{minipage}[t]{0.48\textwidth}
			\centering
			\includegraphics[height=5.0cm]{air_FLAASH}	
			\caption{选择FLAASH Atmospheric Correction}
		\end{minipage}
		\begin{minipage}[t]{0.48\textwidth}
			\centering 
			\includegraphics[height=5.0cm]{air_input}
			\caption{进行文件的输入}
		\end{minipage}
	\end{figure}
	
	(2)选择我们经过辐射定标的图像\code{rc.dat},由于我先前已经做过缩放因子(\code{Scale Factor})的选择,因此我们这里将\code{Radiance Scale Factors}设置为\code{Use single scale factor for all bands},并将\code{single scale factor}设置为\code{1.0}。
	
	\begin{figure}[H]
		\centering
		\begin{minipage}[t]{0.48\textwidth}
			\centering
			\includegraphics[height=5.0cm]{air_inputrc}	
			\caption{选择辐射定标后的图像}
		\end{minipage}
		\begin{minipage}[t]{0.48\textwidth}
			\centering 
			\includegraphics[width=6.0cm]{air_use}
			\caption{设置Scaler}
		\end{minipage}
	\end{figure}
	(3)设置文件的存储位置以及FLAASH的工程文件夹。
	
	\begin{figure}[H]
		\centering
		\begin{minipage}[t]{0.48\textwidth}
			\centering
			\includegraphics[height=4.6cm]{air_output}	
			\caption{设置图像存储位置}
		\end{minipage}
		\begin{minipage}[t]{0.48\textwidth}
			\centering 
			\includegraphics[height=4.6cm]{air_output2}
			\caption{设置工作文件夹}
		\end{minipage}
	\end{figure}
	(4)\code{Sensor Type}选择\code{Multispectral}的\code{Landsat-8 OLI}
	
	\begin{figure}[H]
		\centering
		\includegraphics[height=4.6cm]{air_lad}
	\end{figure}
	
	
	\begin{shaded}
		
		
		\begin{center}\code{Ground Elevation}的获取方法
		\end{center}
		
		
		
		$\triangleright$ 在ENVI菜单栏上点击\code{File},选择\code{Open World Data},再选择\code{Elevation(GMTED2010)},此时加载\code{GMTED2010.jp2
		}出来,在右边\code{Toolbox}中选择\code{Statistics}中的\code{Compute Statistics}。
		
		\begin{figure}[H]
			\centering
			\begin{minipage}[t]{0.48\textwidth}
				\centering
				\includegraphics[height=4.6cm]{gro}	
				\caption{选择Elevation(GMTED2010)}
			\end{minipage}
			\begin{minipage}[t]{0.48\textwidth}
				\centering 
				\includegraphics[height=4.6cm]{gro2}
				\caption{打开Compute Statistics工具}
			\end{minipage}
		\end{figure}
		
		$\triangleright$ 在弹出来的界面选择\code{GMTED2010.jp2},并点击\code{Stats Subset}。
		
		\begin{figure}[H]
			\centering
			\begin{minipage}[t]{0.48\textwidth}
				\centering
				\includegraphics[height=4.6cm]{gro3}	
				\caption{选择GMTED2010.jp2}
			\end{minipage}
			\begin{minipage}[t]{0.48\textwidth}
				\centering 
				\includegraphics[height=4.6cm]{gro4}
				\caption{Stats Subset}
			\end{minipage}
		\end{figure}
		
		$\triangleright$ 在\code{Select Statistics Subset}界面点击\code{File},并选择辐射定标的图像\code{rc.dat}。
		
		\begin{figure}[H]
			\centering
			\begin{minipage}[t]{0.48\textwidth}
				\centering
				\includegraphics[height=4.6cm]{gro5}	
				
			\end{minipage}
			\begin{minipage}[t]{0.48\textwidth}
				\centering 
				\includegraphics[height=4.6cm]{gro6}
				
			\end{minipage}
		\end{figure}
		
		$\triangleright$ 在\code{Compute Statistics Parameters}界面勾选\code{Basic Stats
		},点击\code{OK},算得图像范围内的平均高差为\code{246.356293 m}。
		\begin{figure}[H]
			\centering
			\begin{minipage}[t]{0.48\textwidth}
				\centering
				\includegraphics[height=4.6cm]{gro7}	
				
			\end{minipage}
			\begin{minipage}[t]{0.48\textwidth}
				\centering 
				\includegraphics[height=4.6cm]{gro8}
				
			\end{minipage}
		\end{figure}
		
	\end{shaded}
	
	(5)回到\code{FLAASH}的设置页面,我们将\code{Ground Elevation}填上我们计算得到的高差值.
	\begin{figure}[H]
		\centering
		\includegraphics[height=3.6cm]{air_ge}
	\end{figure}
	(6)右键图层上的\code{rc.dat},点击\code{View Metadata},在弹出的窗口,点击\code{Time}查看时间,并与\code{FLAASH}设置窗口上的时间进行校正。
	\begin{figure}[H]
		\centering
		\begin{minipage}[t]{0.48\textwidth}
			\centering
			\includegraphics[height=4.6cm]{air_viewmeta}	
			
		\end{minipage}
		\begin{minipage}[t]{0.48\textwidth}
			\centering 
			\includegraphics[height=4.6cm]{air_time2}
			
		\end{minipage}
	\end{figure}
	
	(7)设置\code{Atmospheric Model},此时需要查看ENVI的帮助文档。点击ENVI界面上部分的\code{Help}中的\code{Contents}。在帮助文档上输入\code{FLAASH}。
	
	\begin{figure}[H]
		\centering
		\begin{minipage}[t]{0.48\textwidth}
			\centering
			\includegraphics[height=3.6cm]{air_HELP}	
		\end{minipage}
		\begin{minipage}[t]{0.48\textwidth}
			\centering 
			\includegraphics[height=3.6cm]{air_ENVIHELP}
		\end{minipage}
	\end{figure}
	
	(8)进入\code{FLAASH}帮助文档中查看气溶胶模型的时间-模型对应表,由宣城的经纬度(30.940718 \quad N. 经度:118.758816\quad E,低纬度、中纬度皆可)以及时间(10月31日,近似为11月),查到选用的模型为\code{Mid-Latitude Summer}。因此设置\code{Atmospheric Model}为\code{Mid-Latitude Summer}。
	
	\begin{figure}[H]
		\centering
		\begin{minipage}[t]{0.48\textwidth}
			\centering
			\includegraphics[height=4.6cm]{air_ENVIHELP3}	
		\end{minipage}
		\begin{minipage}[t]{0.48\textwidth}
			\centering 
			\includegraphics[width=4.6cm]{air_ENVIHELP4}
		\end{minipage}
	\end{figure}
	
	(9)因为宣城大部分地区为丘陵和乡村,因此设置\code{Aerosol Model
	}为\code{Rural},同时因云量极小,设置\code{Initial Visibility}为\code{40.00}。
	\begin{figure}[H]
		\centering
		\begin{minipage}[t]{0.48\textwidth}
			\centering
			\includegraphics[width=4.6cm]{air_aero}	
		\end{minipage}
		\begin{minipage}[t]{0.48\textwidth}
			\centering 
			\includegraphics[width=4.6cm]{air_iv}
		\end{minipage}
	\end{figure}
	
	(10)点击\code{Multispectral Settings...},点击\code{Kaufman-Tanre Aerosol Retrieval
	}进行设置。
	\begin{figure}[H]
		\centering
		\begin{minipage}[t]{0.48\textwidth}
			\centering
			\includegraphics[height=4.6cm]{air_ms}	
		\end{minipage}
		\begin{minipage}[t]{0.48\textwidth}
			\centering 
			\includegraphics[height=4.6cm]{air_ms1}
		\end{minipage}
	\end{figure}
	
	(11)在\code{ssign Default Values Based on Eetrieval Conditions}点击\code{Defaults->},选择\code{Over-Land Retrieval standard (660:2100 nm)},此时会加载默认设置。
	\begin{figure}[H]
		\centering
		\begin{minipage}[t]{0.48\textwidth}
			\centering
			\includegraphics[height=4.6cm]{air_ms2}	
		\end{minipage}
		\begin{minipage}[t]{0.48\textwidth}
			\centering 
			\includegraphics[height=4.6cm]{air_ms3}
		\end{minipage}
	\end{figure}
	(11)退回原界面,点击\code{Advanced Settings...},查看各个参数的设定,默认值均可接受,因此不做改变。
	\begin{figure}[H]
		\centering
		\begin{minipage}[t]{0.48\textwidth}
			\centering
			\includegraphics[height=4.6cm]{air_as}	
		\end{minipage}
		\begin{minipage}[t]{0.48\textwidth}
			\centering 
			\includegraphics[height=4.6cm]{air_as2}
		\end{minipage}
	\end{figure}
	(12)退回原界面,点击\code{Apply},进行生成。
	\begin{figure}[H]
		\centering
		\begin{minipage}[t]{0.48\textwidth}
			\centering
			\includegraphics[height=4.6cm]{air_apply}	
		\end{minipage}
		\begin{minipage}[t]{0.48\textwidth}
			\centering 
			\includegraphics[height=4.6cm]{air_ing}
		\end{minipage}
	\end{figure}
	
	(13)生成完成后,会得到\code{Result}窗口,绝对大气校正的图像已生成。
	\begin{figure}[H]
		\centering
		\includegraphics[width=4.6cm]{air_result}
	\end{figure}
	\begin{figure}[H]
		\centering
		\includegraphics[height=5cm]{air_flaret}
	\end{figure}
	
	
	\subsubsection{数据对比}
	
	(1)打开校正前校正后的图像,右键图层中的校正后的图像,选择\code{Profiles}中的\code{Spectral}
	\begin{figure}[H]
		\centering
		\includegraphics[height=4.6cm]{com_spe}
	\end{figure}
	(2)选择图像中的水系进行对比。
	
	\begin{figure}[H]
		\centering
		\includegraphics[height=5.2cm]{com_water}
	\end{figure}
	\begin{shaded}
		我们可以观察到,水体在进行大气校正后,蓝绿光波段的值明显变大。
	\end{shaded}
	(3)选择图像中的植被进行对比。
	
	\begin{figure}[H]
		\centering
		\includegraphics[height=5.2cm]{com_veg}
	\end{figure}
	\begin{shaded}
		植被在进行大气校正后,近红外波段的值显著偏大,而其他波段的值有所下降。
	\end{shaded}
	
	
	(4)选择图像中的裸露土进行对比。
	
	\begin{figure}[H]
		\centering
		\includegraphics[height=5.2cm]{com_lan}
	\end{figure}
	\begin{shaded}
		我们可以观察到,土壤在进行大气校正后,数值相对校正值更加平滑。
	\end{shaded}
	
\end{document}